\subsection{Modulation}
\begin{enumerate}[label=\thesubsection.\arabic*.,ref=\thesubsection.\theenumi]
\numberwithin{equation}{enumi}
\numberwithin{figure}{enumi}



\item See Fig. \ref{fig:ee18btech11012_fig1} for the constellation diagram.  The transmitted symbol set is given by 
\begin{align}
\vec{s}_m = \myvec{\cos \frac{2m\pi}{8}\\ \sin \frac{2m\pi}{8}}, \quad m \in \cbrak{0, 1, \dots, 3}.
%\textbf{y = s + n}
\end{align}
%
The numerical values for $\vec{s}_m$ are listed in Table \ref{table: }
\begin{figure}[!ht]
                \resizebox{\columnwidth}{!}{
\begin{tikzpicture}

\draw[<->,thick] (-4,0)--(4,0) node[right]{$\phi_1$};
\draw[<->,thick] (0,-4)--(0,4) node[above]{$\phi_2$};

\filldraw[black] (2.5,0) circle (2pt) node[below] {$s_0$} ;
\filldraw[black] (0,2.5) circle (2pt) node[left] {$s_1$} ;
\filldraw[black] (-2.5,0) circle (2pt) node[below] {$s_2$} ;
\filldraw[black] (0,-2.5) circle (2pt) node[left] {$s_3$} ;

\end{tikzpicture}
}
\label{fig:ee18btech11041_fig1}
\caption{constellation diagram}
\end{figure}

%
\item See Table \ref{table: } for the encoding scheme.
\end{enumerate}
\subsection{Demodulation}
\begin{enumerate}[label=\thesubsection.\arabic*.,ref=\thesubsection.\theenumi]
\item The received symbol is then obtained as
\begin{align}
\vec{y} = \sqrt{E_s}\vec{s} + \vec{n}
\end{align}
%
where $E_s$ is the symbol energy and 

\begin{align}
\vec{n} &\sim \mathcal{N}\brak{\vec{0},\frac{N_0}{2}\vec{I}}
\\
\vec{s} &\in \cbrak{\vec{s}_m}_{m = 0}^{3}
\end{align}
\item The decision rule is given by Fig. \ref{fig:ee18btech11012_fig2} and can be expressed as







\begin{figure}[!ht]

                \resizebox{\columnwidth}{!}{
\begin{tikzpicture}

\draw[<->,thick] (-4,0)--(4,0) node[right]{$y1$};
\draw[<->,thick] (0,-4)--(0,4) node[above]{$y2$};
\draw [dashed]   (-4,-4)--(4,4);
\draw [dashed]   (-4,4)--(4,-4);

\filldraw[black] (2.5,0) circle (2pt) node[below] {$s_0$} node[above] {00};
\filldraw[black] (0,2.5) circle (2pt) node[left] {$s_1$} node[right] {01};
\filldraw[black] (-2.5,0) circle (2pt) node[below] {$s_2$} node[above] {11};
\filldraw[black] (0,-2.5) circle (2pt) node[left] {$s_3$} node [right] {10} ;

\foreach \coordinate/\label/\pos in {{(3,1)/D1/above left},{(1.5,3)/D2/right},{(-3,-2)/D3/above right},{(-1.5,-3)/D4/above right}} \node[\pos] at \coordinate {\label};

\end{tikzpicture}
}

\label{fig:ee18btech11041_fig2}
\caption{decision regions}
	
\end{figure}

Let \textbf{r} be the received bits, \textbf{r} = [$r_1$,$r_2$]. 
\begin{align}
    r_1 = 
    \begin{cases}
    0, &  \textbf{y} \in D1\cup D2 \Longleftrightarrow  y_1 + y_2 > 0\\
    1, &  \textbf{y} \in D3\cup D4 \Longleftrightarrow  y_1 + y_2 < 0
    \end{cases}
    \label{eq:ee18btech11041_eq1}
\end{align}

\begin{align}
    r_2 = 
    \begin{cases}
    0, &  \textbf{y} \in D1\cup D4 \Longleftrightarrow  y_2 - y_1 < 0\\
    1, &  \textbf{y} \in D2\cup D3 \Longleftrightarrow  y_2 - y_1 > 0
    \end{cases}
    \label{eq:ee18btech11041_eq2}
\end{align}

From eq.\ref{eq:ee18btech11041_eq1} and eq.\ref{eq:ee18btech11041_eq2}
\\
For detecting $s_0$, $y_1 > -y_2$ and $y_1 > y_2$.
\\
For detecting $s_1$, $y_1>-y_2$ and $y_1<y_2$.
\\
For detecting $s_2$,  $y_1<-y_2$ and $y_1<y_2$.
\\
For detecting $s_3$, $y_1<-y_2$ and $y_1>y_2$.




\item The following code has simulation of QPSk.
\begin{lstlisting}
 codes/qpsk.py
\end{lstlisting}

\end{enumerate}
