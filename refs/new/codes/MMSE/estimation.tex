%\begin{enumerate}[label=\thesubsection.\arabic*.,ref=\thesubsection.\theenumi]
\numberwithin{equation}{subsection}
\numberwithin{figure}{subsection}
\numberwithin{table}{subsection}

%\item 
The multipath fading cahnnel is modelled as a linear finite impulse-reponse filter.
Let $\vec{x}_n \in \cbrak{\vec{s}_k}_{0}^{7}$ be transmitted 8-PSK symbols.  The received symbols can then be expressed as
\begin{align}
\vec{r}_n = \vec{x}_n * \vec{g}_n
\end{align}
where
%\\
%Let $s_i$ denote the set of samples at the input to the channel, Then samples $Rk_i$ at the output of the channel are related to $s_i$ through:
%\begin{align}
%Rk_i=\sum_{n=-N_1}^{N2}s_{i-n}g_n
%\end{align}
%Where $g_n$ is the set of tap weights given by:
\begin{align}
\label{eq:channel_taps}
\vec{g}_n &= 
%\vec{a}_n*h_n
\sum_{k=0}^{K-1}\vec{a}_k \sinc{\brak{\frac{\tau_k}{t_s}-n}}
\\
k &= 0, 1, \dots K-1
\\
-N_1 &\leq n \leq N_2
\end{align}
%and 
%\begin{align}
%\end{align}
and
\begin{enumerate}
\item $t_s$ is the input sample period to the channel
\item $\tau_k$ where $1\leq k \leq K$is the set of path delays(pd).
\item K is the total number of paths in the multiple fading channel. Here, K=5 
\item $a_k$ where $1\leq k \leq K$is the set of complex path gains (pg).
$N_1$ and $N_2$ are chosen so that $g_n$ is small when n is less than$-N_1$ and greater than $N_2$.In the given code,
\begin{align}
N_1=N_2=800
\end{align}
\end{enumerate}
\eqref{eq:channel_taps} can be expressed as
\begin{align}
\label{eq:channel_taps_mat}
\vec{g}=
%\vec{g}_n &= 
%\vec{a}_n*h_n
\vec{T}\vec{a}
\end{align}
where $\vec{T}$ is a $\brak{N_1+N_2+1}\times K$ matrix with entries
\begin{align}
T_{ij} = \sinc{\brak{\frac{\tau_j}{t_s}-i}}
\\
j &= 0, 1, \dots K-1
\\
-N_1 &\leq i \leq N_2
\end{align}
and $\vec{a}$ is a $\vec{K}\times$ vector with entries $a_k$.
