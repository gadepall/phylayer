MMSE require estimation of the SNR.  The signal power is estimated using $\vec{h}$ in the pilot duration.  The noise power is
estimated in the rampup time in Fig. \ref{fig:ee18btech11041_frame}.

\begin{enumerate}[label=\thesubsection.\arabic*.,ref=\thesubsection.\theenumi]

\numberwithin{equation}{enumi}
\numberwithin{figure}{enumi}
\numberwithin{table}{enumi}

\item Let $\vec{h}_p$
%
be the channel estimate using the $p$th pilot block.  Then, the average symbol SNR at the receiver is computed using
\begin{align}
\hat{E}_s = \frac{1}{2\floor*{\frac{P_f}{P}}}\sum_{p=1}^{\floor*{\frac{P_f}{P}}}\norm{\vec{h}_p}^2
\end{align}
%
were $P_f$ is the total number of pilot symbols in a frame.
\item The noise variance is computed as
\begin{align}
\hat{\sigma}^2 = \frac{1}{2K}\sum_{i=1}^{K}\abs{Y_k}^2
\end{align}
%
where $K$ is the maximum number of symbols that could possibly be transmitted during rampup time.
\item From the above, 
\begin{align}
SNR = \frac{\hat{E}_s}{\hat{\sigma}^2}
\end{align}
%$P=10$ pilot symbols are used at a time for channel estimation.  The channel filter has length $L=5$.  See Table \ref{table:ChannelParams} for  details.
%The consequent model is
%\begin{align}
%\vec{y}_p = \vec{x}_p*\vec{h}+ \vec{n}_p
%\label{eq:mmse_chan_mod}
%\end{align}
%\begin{table}[!h]
%\centering
%\input{./tables/MMSE/ChannelParams.tex}
%\caption{}
%\label{table:ChannelParams}
%\end{table}
%\item Let
%\begin{align}
%\vec{x}_p \fourier \vec{X}_p, 
%\vec{y}_p \fourier \vec{Y}_p, 
%\vec{h} \fourier \vec{H}_p, 
%\end{align} 
%be the DFTs of the signals.  Then, 
%\begin{align}
%\vec{H}_p = \frac{\vec{Y}_p}{\vec{X}_p}
%\\
%\text{and }\vec{H}_p \fourier \vec{h}
%\label{eq:mmse_chan_mod_dft}
%\end{align} 
%This is how channel estimation is done $\because$ both $\vec{x}_p$
%and $\vec{y}_p$ are known at the receiver.
%\item While \eqref{eq:mmse_chan_mod_dft} cannot be applied directly, since the lengths of $\vec{x}_p (P)$  and $\vec{h}(L)$ are different, resulting in a circular convolution.  To address this, we do the following operations
%\begin{multline}
%\vec{y} = \myvec{\vec{I}_{P} & \vec{0}_{P\times L-1}}\lsbrak{
%\brak{\vec{I}+\vec{R}}\myvec{\vec{1}_{L-1}^T   & \vec{0}_{P}^T }
%}
%\\
%+\rsbrak{
%\myvec{\vec{0}_{L-1}^T & \vec{1}_{P-L+1}^T  & \vec{0}_{L-1}^T}\vec{y}_p}
%\end{multline} 
%where
%\begin{align}
%\vec{R} = \myvec
%{
%0 & 0 & 0 & \dots & 1
%\\
%0 & 0 & \dots & 1 & 0
%\\
% & &\vdots & & 
%\\
%1 & 0 & \dots & 0 & 0
%}
%\label{eq:mmse_chan_mod_reflect}
%\end{align} 
%is a reflection matrix.
%The channel is now estiamted as
%\begin{align}
%\vec{y} \fourier \vec{Y}
%\\
%\frac{\vec{Y}}{\vec{X}_p} \fourier \vec{h}
%\label{eq:mmse_chan_mod_fft}
%\end{align} 
\end{enumerate}
