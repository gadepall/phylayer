The pilot symbols are used for channel estimation.
\begin{enumerate}[label=\thesubsection.\arabic*.,ref=\thesubsection.\theenumi]

\numberwithin{equation}{enumi}
\numberwithin{figure}{enumi}
\numberwithin{table}{enumi}

\item $P=10$ pilot symbols are used at a time for channel estimation.  The channel filter has length $L=5$.  See Table \ref{table:ChannelParams} for  details.
The consequent model is
\begin{align}
\vec{y}_p = \vec{x}_p*\vec{h}+ \vec{n}_p
\label{eq:mmse_chan_mod}
\end{align}
\begin{table}[!h]
\centering
\input{./tables/MMSE/ChannelParams.tex}
\caption{}
\label{table:ChannelParams}
\end{table}
\item Let
\begin{align}
\vec{x}_p \fourier \vec{X}_p, 
\vec{y}_p \fourier \vec{Y}_p, 
\vec{h} \fourier \vec{H}_p, 
\end{align} 
be the DFTs of the signals.  Then, 
\begin{align}
\vec{H}_p = \frac{\vec{Y}_p}{\vec{X}_p}
\\
\text{and }\vec{H}_p \fourier \vec{h}
\label{eq:mmse_chan_mod_dft}
\end{align} 
This is how channel estimation is done $\because$ both $\vec{x}_p$
and $\vec{y}_p$ are known at the receiver.
\item While \eqref{eq:mmse_chan_mod_dft} cannot be applied directly, since the lengths of $\vec{x}_p (P)$  and $\vec{h}(L)$ are different, resulting in a circular convolution.  To address this, we do the following operations
\begin{multline}
\vec{y} = \myvec{\vec{I}_{P} & \vec{0}_{P\times L-1}}\lsbrak{
\brak{\vec{I}+\vec{R}}\myvec{\vec{1}_{L-1}^T   & \vec{0}_{P}^T }
}
\\
+\rsbrak{
\myvec{\vec{0}_{L-1}^T & \vec{1}_{P-L+1}^T  & \vec{0}_{L-1}^T}\vec{y}_p}
\end{multline} 
where
\begin{align}
\vec{R} = \myvec
{
0 & 0 & 0 & \dots & 1
\\
0 & 0 & \dots & 1 & 0
\\
 & &\vdots & & 
\\
1 & 0 & \dots & 0 & 0
}
\label{eq:mmse_chan_mod_reflect}
\end{align} 
is a reflection matrix.
The channel is now estiamted as
\begin{align}
\vec{y} \fourier \vec{Y}
\\
\frac{\vec{Y}}{\vec{X}_p} \fourier \vec{h}
\label{eq:mmse_chan_mod_fft}
\end{align} 
\end{enumerate}
